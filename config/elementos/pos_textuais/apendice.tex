

% Apêndice A
\chapter{Demonstrações} \label{apendice_demonstracoes}

% proposição 1
\begin{proposition}
  \label{proposicao1}

  Sendo $Y$ a variável independente de um modelo log-linear do tipo
  
  \begin{equation}
    \log(Y) = \beta_0 + \beta_1X_1 + \beta_2X_2 + \ldots + \beta_pX_p
  \end{equation} \label{eq:loglinear}
  
  \noindent o coeficiente $\beta_i$ relacionado a um preditor contínuo representa a variação relativa em $Y$ dada a variação unitária em $X_i$.

\end{proposition}

\begin{proof}
  Considere a seguinte equação de regressão log-linear simples:
  \begin{equation} \label{eq:loglinearsimples}
    \log(Y) = \beta_0 + \beta_1X_1
  \end{equation}

  Derivando ambos lados de \eqref{eq:loglinearsimples} em relação a $X_1$, temos

  \begin{equation} \label{eq:derivada1}
    \frac{\partial \log(Y)}{\partial X_1} = \beta_1
  \end{equation}

  Pela regra da cadeia, temos

  $$
  \begin{aligned}
    \frac{1}{Y}Y' &= \beta_1\\
    \frac{Y'}{Y} &= \beta_1
  \end{aligned}
  $$

  Logo, $\beta_1$ representa a variação relativa em $Y$ dada a variação unitária em $X_1$. Analogamente, $\beta_p$ representa a variação relativa em $Y$ dada a variação unitária em $X_p$.

\end{proof}

% proposição 2
\begin{proposition}
  \label{proposicao2}

  Sendo $Y$ a variável dependente de um modelo log-linear do tipo
  
  \begin{equation}
    \log(Y) = \beta_0 + \beta_1X_1 + \beta_2X_2 + \ldots + \beta_pX_p
  \end{equation} \label{eq:loglinear2}
  
  \noindent $e^{\beta_i}$, com $\beta_i$ relacionado a um preditor categórico, representa a variação relativa em $Y$ dada a presença do nível $i$ de $X_i$.
\end{proposition}

\begin{proof}
  Mantém-se a coerência ao exponenciar ambos os lados de \eqref{eq:loglinear2}:

  $$
  \begin{aligned}
    e^{ln(Y)} &= e^{\beta_0 + \beta_1X_1} \\
    Y &= e^{\beta_0}e^{\beta_1X_1}
  \end{aligned}
  $$

  Então,

  $$
  Y = 
  \begin{cases}
    e^{\beta_0} & \text{se $X_1=0$} \\
    e^{\beta_0}e^{\beta_1} & \text{se $X_1=1$}
  \end{cases}
  $$

  Logo, $e^{\beta_1}$ representa a variação relativa em $Y$ dada a presença do nível $i$ de $X_i$. Analogamente, $e^{\beta_p}$ representa a variação relativa em $Y$ dada a variação unitária em $X_p$.
\end{proof}

% proposição 3
\begin{proposition}
  \label{proposicao3}

  Sendo $Y$ a variável dependente de um modelo log-linear do tipo
  
  \begin{equation}
    \log(Y) = \beta_0 + \beta_1X_1 + \beta_2X^2_1 + \ldots + \beta_pX_p
  \end{equation} \label{eq:loglinear3}
  
  \noindent $\beta_1 + 2\beta_2X_1$, com $\beta_1$ e $\beta_2$ relacionados a um preditor contínuo, representa a variação relativa em $Y$ em relação a $X_1$, no ponto $X_2 = x_2$.
\end{proposition}

\begin{proof}
  Derivando ambos lados de \eqref{eq:loglinear3} em relação a $X_1$, temos

  \begin{equation} \label{eq:derivada1}
    \frac{\partial \log(Y)}{\partial X_1} = \beta_1 + 2\beta_2X_1
  \end{equation}

  Pela regra da cadeia, temos

  $$
  \begin{aligned}
    \frac{1}{Y}Y' &= \beta_1 + 2\beta_2X_1\\
    \frac{Y'}{Y} &= \beta_1 + 2\beta_2X_1
  \end{aligned}
  $$

  Portanto, $\beta_1 + 2 \beta_2X_1$ representa a variação relativa em $Y$ dada variação unitária em $X_1$ no ponto $X_2 = x_2$.
\end{proof}
