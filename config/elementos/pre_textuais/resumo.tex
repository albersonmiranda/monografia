% RESUMO - PT
\begin{resumo}
  \vspace{-15pt}
  
  Este estudo analisa os efeitos da escolaridade na determinação da renda dos trabalhadores formais no estado do Espírito Santo, considerando variáveis como gênero, raça e experiência. Utilizando dados da Relação Anual de Informações Sociais (Rais) de 2006 e 2022, aplica-se uma regressão linear múltipla para identificar padrões e tendências ao longo do tempo. Os resultados indicam uma redução no impacto da escolaridade sobre a renda, além de evidenciar desigualdades raciais e de gênero persistentes no mercado de trabalho formal. No segundo capítulo, argumenta-se que o sistema educacional, enquanto inserido na superestrutura capitalista, atua como um instrumento de reprodução das relações sociais de poder. A análise histórica e teórica mostra que a educação serviu à normalização moral e à perpetuação das estruturas de poder vigentes. A educação matemática crítica, embora busque romper com essas estruturas, enfrenta limitações significativas devido à sua inserção em instituições panópticas que exercem micro-poderes sobre os indivíduos.

  Palavras-chave: \palavraschaveemlinha
\end{resumo}


% RESUMO - EN
\begin{resumo}[Abstract]
  \vspace{-15pt}
  
  \begin{otherlanguage*}{english}
    This study examines the effects of education on the determination of income among formal workers in the state of Espírito Santo, considering variables such as gender, race, and experience. Utilizing data from the Annual Social Information Relationship (Rais) of 2006 and 2022, a multiple linear regression is applied to identify patterns and trends over time. The results indicate a reduction in the impact of education on income, alongside persistent racial and gender inequalities in the formal labor market. In the second chapter, it is argued that the educational system, while embedded in the capitalist superstructure, acts as an instrument for reproducing social power relations. The historical and theoretical analysis shows that education has served the moral normalization and perpetuation of existing power structures. Critical mathematics education, although seeking to break with these structures, faces significant limitations due to its insertion in panoptic institutions that exercise micro-powers over individuals.
  
  Keywords: \inlinekeywords
\end{otherlanguage*}
\end{resumo}