% elementos pré-textuais 

% título do sumário
\ifdefined\contentsname
  \renewcommand*\contentsname{SUMÁRIO}
\else
  \newcommand\contentsname{SUMÁRIO}
\fi

% capa 
\imprimircapa

% folha de rosto 
% o * indica que haverá a ficha bibliográfica 
\imprimirfolhaderosto*

% ficha catalográfica 
%\begin{fichacatalografica}
%  \includepdf{config/elementos/ficha_ufes.pdf}
%\end{fichacatalografica}


% substituir pela ficha em pdf fornecida pela UFES após defesa 
\begin{fichacatalografica}
	\sffamily
	\vspace*{\fill}					% Posição vertical
	\begin{center}					% Minipage Centralizado
	\fbox{\begin{minipage}[c][8cm]{15cm}		% Largura
	\small
	\imprimirautor
	
	\hspace{0.5cm} \imprimirtitulo  / \imprimirautor. --
	\imprimirlocal, \imprimirdata-
	
	\hspace{0.5cm} \thelastpage p. : il. (algumas color.) ; 30 cm.\\
	
	\hspace{0.5cm} \imprimirorientadorRotulo~\imprimirorientador\\
	
	\hspace{0.5cm}
	\parbox[t]{\textwidth}{\imprimirtipotrabalho~--~\imprimirinstituicao,
	\imprimirdata.}\\
	
	\hspace{0.5cm}
		1. xxx.
		2. xxx.
		3. xxx.
    4. xxx.
		I. Broetto, Geraldo Cláudio.
		II. Instituto Federal do Espírito Santo.
		III. Coordenadoria de Licenciatura em Matemática.
		IV. Título 			
	\end{minipage}}
	\end{center}
\end{fichacatalografica}

% folha de aprovação Ufes (pdfpages não adiciona assinatura digital, combinar via Adobe)
%\begin{folhadeaprovacao}
%  \includepdf{config/elementos/folha_aprovacao.pdf}
%\end{folhadeaprovacao}

% substituir pela folha assinada pela banca após defesa 
\begin{folhadeaprovacao}

  \begin{center}
    {\ABNTEXchapterfont\large\imprimirautor}

    \vspace*{\fill}\vspace*{\fill}
    \begin{center}
      \ABNTEXchapterfont\bfseries\Large\imprimirtitulo
    \end{center}
    \vspace*{\fill}
    
    \hspace{.45\textwidth}
    \begin{minipage}{.5\textwidth}
        \imprimirpreambulo
        \vspace*{1cm}
        Aprovada em XX de XX de 2024.\\[2cm]
        \textbf{COMISSÃO EXAMINADORA} \\
        \assinatura{\textbf{\imprimirorientador} \\ Instituto Federal do Espírito Santo \\ Orientador}
        \assinatura{\textbf{\imprimircoorientador} \\ Instituto Federal do Espírito Santo \\ Coorientador}
        \assinatura{\textbf{Prof. Dr. Rodolpho Chaves} \\ Instituto Federal do Espírito Santo}
        \assinatura{\textbf{Prof. Dr. José Carlos Thompson da Silva} \\ Instituto Federal do Espírito Santo}
        %\assinatura{\textbf{Professor} \\ Convidado 3}
        %\assinatura{\textbf{Professor} \\ Convidado 4}
    \end{minipage}%
   \end{center}
  
\end{folhadeaprovacao}

%% dedicatória 
%\begin{dedicatoria}
%   \vspace*{\fill}
%   \centering
%   \noindent
%   \textit{Exemplo de dedicatória,\\\lipsum[10].} \vspace*{\fill}
%\end{dedicatoria}

%% agradecimentos 
%\begin{agradecimentos}
%  
%\end{agradecimentos}

% epígrafe 
%\begin{epigrafe}
%    \vspace*{\fill}
%	\begin{flushright}
%		\textit{``Modelo de epígrafe, \\
%		modelo de epígrafe.''}
%	\end{flushright}
%\end{epigrafe}

% resumo 

\setlength{\absparsep}{18pt}
\begin{resumo}
  Este estudo analisa os efeitos da escolaridade na determinação da renda dos trabalhadores formais no estado do Espírito Santo, considerando variáveis como gênero, raça e experiência. Utilizando dados da Relação Anual de Informações Sociais (Rais) de 2006 e 2022, aplica-se uma regressão linear múltipla para identificar padrões e tendências ao longo do tempo. Os resultados indicam uma redução no impacto da escolaridade sobre a renda, além de evidenciar desigualdades raciais e de gênero persistentes no mercado de trabalho formal. No segundo capítulo, argumenta-se que o sistema educacional, enquanto inserido na superestrutura capitalista, atua como um instrumento de reprodução das relações sociais de poder. A análise histórica e teórica mostra que a educação serviu à normalização moral e à perpetuação das estruturas de poder vigentes. A educação matemática crítica, embora busque romper com essas estruturas, enfrenta limitações significativas devido à sua inserção em instituições panópticas que exercem micro-poderes sobre os indivíduos.

  \textbf{Palavras-chave}: Escolaridade, Renda, Desigualdade Social, Educação Matemática, Capitalismo.
\end{resumo}

% abstract 
\begin{resumo}[Abstract]
  \begin{otherlanguage*}{english}
    This study examines the effects of education on the determination of income among formal workers in the state of Espírito Santo, considering variables such as gender, race, and experience. Utilizing data from the Annual Social Information Relationship (Rais) of 2006 and 2022, a multiple linear regression is applied to identify patterns and trends over time. The results indicate a reduction in the impact of education on income, alongside persistent racial and gender inequalities in the formal labor market. In the second chapter, it is argued that the educational system, while embedded in the capitalist superstructure, acts as an instrument for reproducing social power relations. The historical and theoretical analysis shows that education has served the moral normalization and perpetuation of existing power structures. Critical mathematics education, although seeking to break with these structures, faces significant limitations due to its insertion in panoptic institutions that exercise micro-powers over individuals.

    \textbf{Keywords}: Education, Income, Social Inequality, Mathematics Education, Capitalism.
  \end{otherlanguage*}
\end{resumo}

% lista de ilustrações 
\pdfbookmark[0]{\listfigurename}{lof}
\listoffigures*
\cleardoublepage

% lista de quadros 
%\pdfbookmark[0]{\listofquadrosname}{loq}
%\listofquadros*
%\cleardoublepage

% lista de tabelas 
\pdfbookmark[0]{\listtablename}{lot}
\listoftables*
\cleardoublepage

% lista de abreviaturas 
\begin{siglas}
  \item[IPCA] Índice de Preços ao Consumidor Amplo
  \item[PCN] Parâmetros Curriculares Nacionais
  \item[Rais] Relação Anual de Informações Sociais
  \item[LDB] Lei de Diretrizes e Bases da Educação Nacional
  \item[K-8] Jardim de infância (do inglês, \textit{kindergarten}) até o 8º ano do ensino fundamental
\end{siglas}

% lista de símbolos 
\begin{simbolos}
  \item[$R^2$] Coeficiente de determinação
  \item[$\beta$] Coeficiente de regressão
  \item[$\mu$] Erro estocástico
  \item[$exp$] Experiência
  \item[$Vr_{i:j}$] Valor real deflacionado no período $i$ na data-base $j$
  \item[$I_i$] Índice de preços fixado na data-base $j$
  \item[$V_i$] Valor nominal no período $i$
\end{simbolos}

% sumário 
\pdfbookmark[0]{\contentsname}{toc}
\tableofcontents*
\cleardoublepage

% elementos textuais 
\textual
\pagestyle{simple}